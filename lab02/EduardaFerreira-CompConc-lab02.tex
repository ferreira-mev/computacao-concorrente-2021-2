% !TEX program = xelatex

\documentclass{article}

\input{/home/duda/Documents/tex/examheaders.tex}

\usepackage{enumitem}
\usepackage{amsmath}
\usepackage{xurl}  % linebreaks at any character

\usepackage{graphicx}
\graphicspath{ {..} }

\usepackage{hyperref}

\hypersetup{
    colorlinks=true,
    linkcolor=gray,
    filecolor=magenta,      
    urlcolor=blue
}

\urlstyle{same}

\usepackage{listings}

\lstset{
  basicstyle=\ttfamily,
  breaklines=true,
  breakatwhitespace=false
} 

\begin{document}

\begin{large}
    \compconc{Laboratório 2}
\end{large}

O trabalho foi feito numa máquina Linux (Fedora 35 Workstation) com 4 cores físicos e 8 lógicos, devido ao hyperthreading, conforme parte da saída do comando \texttt{lscpu}, reproduzida abaixo:

\begin{lstlisting}
Architecture:            x86_64
  CPU op-mode(s):        32-bit, 64-bit
    ...
CPU(s):                  8
  On-line CPU(s) list:   0-7
Vendor ID:               GenuineIntel
  Model name:            Intel(R) Core(TM) i7-7700HQ CPU @ 2.80GHz
    CPU family:          6
    Model:               158
    Thread(s) per core:  2
    Core(s) per socket:  4
    Socket(s):           1
    Stepping:            9
    CPU max MHz:         3800.0000
    CPU min MHz:         800.0000
    ...
\end{lstlisting}

\end{document}