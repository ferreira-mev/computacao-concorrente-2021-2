% !TEX program = xelatex

\documentclass{article}

\input{/home/duda/Documents/tex/examheaders.tex}

\usepackage{enumitem}
\usepackage{amsmath}
\usepackage{xurl}  % linebreaks at any character

\usepackage{graphicx}
\graphicspath{ {..} }

\usepackage{hyperref}

\hypersetup{
    colorlinks=true,
    linkcolor=gray,
    filecolor=magenta,      
    urlcolor=blue
}

\urlstyle{same}

\usepackage{listings}

\lstset{
  basicstyle=\ttfamily,
  breaklines=true,
  breakatwhitespace=false
} 

\usepackage{csvsimple}

\begin{document}

\begin{large}
    \compconc{Laboratório 2}
\end{large}

O trabalho foi feito numa máquina Linux (Fedora 35 Workstation) com 4 cores físicos e 8 lógicos, devido ao hyperthreading, conforme parte da saída do comando \texttt{lscpu}, reproduzida abaixo:

\begin{lstlisting}
Architecture:            x86_64
  CPU op-mode(s):        32-bit, 64-bit
    ...
CPU(s):                  8
  On-line CPU(s) list:   0-7
Vendor ID:               GenuineIntel
  Model name:            Intel(R) Core(TM) i7-7700HQ CPU @ 2.80GHz
    CPU family:          6
    Model:               158
    Thread(s) per core:  2
    Core(s) per socket:  4
    Socket(s):           1
    Stepping:            9
    CPU max MHz:         3800.0000
    CPU min MHz:         800.0000
    ...
\end{lstlisting}

A tabela com as medições feitas conforme o enunciado do laboratório está a seguir.

\begin{center}
  \begin{tabular}{c}
    \csvautotabular{perf_gains.csv}
    % Não sou familiarizada com esse pacote csvsimple, acabei de descobrir que ele existe. Acho que não fiz muito direito não, não estou em paz com esse {c} único, mas por ora serve; LaTeX, como CSS, é algo de caminhos misteriosos e paradigma orientado à gambiarra.
  \end{tabular}
\end{center}

Não resisti a incluir 8 threads nas medidas, já que -- supostamente -- meu processador tem até essa capacidade. Foi interessante observar que, provavelmente como consequência de a quantidade de cores \emph{físicos} ser 4, usar 8 threads levou a uma performance menor do que 4, apesar de ainda ter sido superior ao uso de 2. Até esse ponto, dobrar o número de threads tinha aproximadamente dobrado a razão ${T_{\text{sequencial}}}/{T_{\text{concorrente}}}$; a partir daí, acredito que o processador esteja, na verdade, simulando um programa paralelo com execução sequencial, o que gera um overhead sem um ganho tão grande de performance.

Um fato que achei curioso foi a execução do programa concorrente com apenas uma thread ter sido marginalmente mais rápida do que a do programa sequencial. Eu esperava que, nesse caso, a versão concorrente fosse mais lenta, devido ao overhead de criação de uma thread sem ganho de eficiência devido à paralelização. Não sei se é alguma otimização ou da própria linguagem, ou acidentalmente introduzida por mim pela forma como escrevi o código; quando rodei testes manuais com valores menores, tenho a impressão de ter observado o contrário: programas concorrentes ligeiramente mais lentos (creio que algo como 0.9) para matrizes pequenas.

\end{document}