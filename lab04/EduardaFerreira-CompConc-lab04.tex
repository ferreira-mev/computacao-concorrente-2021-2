% !TEX program = xelatex

\documentclass{article}

\input{/home/duda/Documents/tex/examheaders.tex}

\usepackage{enumitem}
\usepackage{amsmath}
\usepackage{xurl}  % linebreaks at any character

\usepackage{graphicx}
\graphicspath{ {..} }

\usepackage{hyperref}

\hypersetup{
    colorlinks=true,
    linkcolor=gray,
    filecolor=magenta,      
    urlcolor=blue
}

\urlstyle{same}

\usepackage{listings}

\lstset{
  basicstyle=\ttfamily,
  breaklines=true,
  breakatwhitespace=false
} 

\begin{document}

\begin{large}
    \compconc{Laboratório 4}
\end{large}

As sequências de execução serão identificadas pela ordem em que as instruções são executadas. Será sempre presumido que a linha \texttt{int x=0;} é executada primeiro. Cada linha será identificado pelo número da thread, seguido do número da linha propriamente dita, constante no pdf. Serão incluídas apenas os comandos executados até a impressão da saída desejada, ainda que a execução do programa continue depois disso. Exceto quando explicitado o contrário, quando uma linha é listada, é presumido que as instruções de máquina correspondentes a ela sejam executadas completamente antes da execução da linha seguinte.

\begin{itemize}
  \item 1: T2.1, T2.2, T3.1, T3.2, T3.3;
  \item -1: T1.1, T1.2, T1.3, T1.4, T1.5 (obs: esse caso corresponde simplesmente ao término da execução da thread 1 antes do início da execução das demais);
  \item 0: T2.1, T2.2, T3.1, T3.2, T1.1, T3.3;
  \item 2: T3.1, T3.2, T2.1, T3.3;
  \item -2: T1.1; execução simultânea de T1.2 e T2.1, com cada uma delas executando a instrução de máquina que acessa o valor de \texttt{x} quando este vale -1, de modo que cada uma, após somar 1, atribuiria a essa variável o valor 0; T1.3, T1.4, T2.2, T1.5;
  \item 3: T3.1, T3.2; execução simultânea de T1.1 e T2.1, com ambas acessando o valor de \texttt{x} quando este é 1, a thread 1 decrementando \texttt{x} para 0, a thread 2 incrementando \texttt{x} para 2, e esta última executando por último a escrita do valor na memória, de modo que \texttt{x} termina com valor 2; T1.2, T3.3;
  \item -3: saída impossível, dado que, para que o valor de \texttt{x} seja impresso, é necessário que este chegue ou à linha T1.4 como -1, ou a T3.2 como 1. No primeiro caso, restaria, no máximo, somente mais um decremento de uma unidade (o da linha T2.2) a ser executado, o que levaria a variável a um valor mínimo de -2, ao passo que, no segundo caso, haveria, no máximo, mais três decrementos de uma unidade a serem executados (T1.1, T1.3, T1.2), o que não permitiria que \texttt{x} assumisse um valor menor do que -2;
  \item 4: saída impossível, pois a variável \texttt{x} é inicializada com valor 0 e há apenas três incrementos de uma unidade, tomando conjuntamente todas as threads;
  \item -4: saída impossível, pois a variável \texttt{x} é inicializada com valor 0 e há apenas três decrementos de uma unidade, tomando conjuntamente todas as threads.
\end{itemize}

\end{document}